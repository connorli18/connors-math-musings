
\section{Different Infinities?}
\subsection{Overview}
Recently, I had a very intelligent friend (who is great at math...hopefully he reads this one day) ask me a question about infinity.
Through his coursework, he'd encountered this concept of $\infty$, or infinity. 
And while this concept may seem pretty obvious to the average eye, he was curious about something he heard in class known as the "different classes" of infinity. \\
\\
Let me explain this without too much higher-level math knowledge.
Since $\infty$ is widely known as an unreachable number (a concept that represents a lot of stuff), how can there be different types of infinity?
If $\infty$ is unquantifiable, how can there be different types of it $\dots$ if you get big enough, you're just at $\infty$, right? \\
\\
Well, not exactly. Let me explain $\dots$
\subsection{An Intuitive Example}
Take the set of integers $\mathbb{Z} = \{\dots,\;-3,\;-2,\;-1,\;0,\;1, \;2,\;3,\;\dots\}$. It should be pretty obvious that the size of this set is infinite since the natural numbers go one forever, continuously increasing by $1$. \\
\\
Now, consider the following one-to-one mapping from the set of integers, $\mathbb{Z}$ to the set of rational numbers $\mathbb{Q}$ between $[-1, 1]$ inclusive. 
What this means is that we will create a one-to-one correspondence (non-repeating) between the integers and the real numbers between $[-1, 1]$. 
If you aren't impressed, you should be! Somehow, we've managed to squeeze every counting number into what seems like such a small interval from $-1$ to $1$.
\\
\\
Let's construct this map properly. First, take $0$ and map it to $0$.
$$
0 \mapsto 0
$$
Notationally, we write $\mapsto$ to denote the mapping from the set of origin (integers) to the set of destination (rational numbers). Now, take every other number in $\mathbb{Z}$ that is not $0$ and map it to it's inverse. Specifically for $n \in \mathbb{Z}$ where $n \neq 0$, we map $n \mapsto \frac{1}{n}$.
$$
1 \mapsto 1 \quad \quad -1 \mapsto -1 \quad \quad 2 \mapsto \frac{1}{2} \quad \quad -2 \mapsto -\frac{1}{2} \quad \quad 3 \mapsto \frac{1}{3} \quad \quad -3 \mapsto -\frac{1}{3} \quad \quad \dots
$$
We now have the following one-to-one correspondence, without repetition, between the integers and the rational numbers between $[-1, 1]$ illustrated as follows.
$$
\begin{tabular}{c c c c c c c c c c c }
    $\mathbb{Z}$: & $\dots$ & $-3$ & $-2$ & $-1$ & $0$ & $1$ & $2$ & $3$ & $\dots$ \\
    $\mathbb{Q}$: & $\dots$ & $-\frac{1}{3}$ & $-\frac{1}{2}$ & $-1$ & $0$ & $1$ & $\frac{1}{2}$ & $\frac{1}{3}$ & $\dots$
\end{tabular}
$$
As you can see, we've managed to map every integer to a rational number between $[-1, 1]$ without any repetition. 
This shows that there are an infinite amount of numbers in the form of $\frac{1}{n}$ where $n \in \mathbb{Z}$ in the interval $[-1, 1]$. 
And since numbers in the form of $\frac{1}{n}$ where $n \in \mathbb{Z}$ are only a subsection of the rational numbers $\in [-1,1]$, it really shouldn't be a question that there are an infinite number of rational numbers in the interval $[-1, 1]$.\\
\\
Now, we all know that all rational numbers aren't contained in the interval $[-1,1]$. This is where we start to see the idea of different types of infinity come into play. This may seem confusing, but try to follow me here. We just proved that there are an infinite number of rational numbers in the interval $[-1, 1]$. Now, if there are infinite intervals with the same properties i.e. $[-3, -1]$, $[-1, 1]$, $[1, 3]$, etc., then there are an infinite number of rational numbers in each of these infinite intervals. It's almost like $\infty^2$. \\
\\
Similarly, we can extend this idea to the set of real numbers $\mathbb{R}$. Take any rational number $q \in \mathbb{Q}$ and start taking roots. For example, $\sqrt{q}$, $\sqrt[3]{q}$, $\sqrt[4]{q}$, etc. You can take any natural root of a rational number, and you will always end up with a real number. This means that for any one rational number, we can use it to create an infinite number of unique real numbers. It's almost like $\infty^3$. \\
\\
If you keep extending this idea, you'll see that there's no end. Infinities can always be raised the power of infinite infinities, and chaos will ensue. So, mathematians have come up with a way to classify these different types of infinity, which I will explain in the next section.
\subsection{A More Formal Approach}
The concepts of countable and uncountable infinity are fundamental in set theory, a branch of mathematics dealing with the nature of infinity and the structure of sets. 
We call a set countably infinite if its elements can be put into a one-to-one correspondence with the natural numbers (as we did above). 
Examples of countably infinite sets include the set of all natural numbers, integers, and rational numbers. Despite being infinite, these sets can be enumerated in a way that each element can eventually be assigned a natural number.\\
\\
In contrast, we call a set uncountably infinite if its elements cannot be listed in such a sequence. 
The most well-known example of an uncountably infinite set is the set of real numbers between any two given real numbers, such as between $0$ and $1$. 
An interesting proof of this is Cantor’s diagonal argument, which shows that no matter how we attempt to list the real numbers, there will always be some numbers left out of the list. 
This indicates that the cardinality (a measure of the size of a set) of the set of real numbers is strictly greater than that of the natural numbers, which then implies a higher level of infinity (uncountable infinity).\\
\\
It's getting late, and I'm getting sleepy - so check \href{https://math.libretexts.org/Under_Construction/Stalled_Project_(Not_under_Active_Development)/Additional_Discrete_Topics_(Dean)/Infinite_Sets_and_Cardinality#:~:text=An%20infinite%20set%20that%20can,with%20N%20is%20uncountably%20infinite.}{this article} out for more information about uncountable/countable sets, infinities, and the like.
