
\section{Logic}
\subsection{Implication}
Logic table for $(p \rightarrow q) \implies$ 
\begin{tabular}{c|c|c}
    $p$ & $q$ & $p \rightarrow q$ \\
     \hline
     T & T & T \\
     T & F & F \\
     F & T & T \\
     F & T & T
\end{tabular} \\
Logic Rule: $p \rightarrow q \equiv \neg p \vee q$ (1)
\begin{itemize}
    \item Inverse: $q \rightarrow p$ (2)
    \item Converse: $\neg p \rightarrow \neg q$ (2)
    \item Contrapositive: $\neg q \rightarrow \neg p$ (1)
\end{itemize}
\subsection{Propositions}
Unary connectives: $\neg$, Binary connectives: $\wedge, \vee, \rightarrow, \iff, \oplus$\\
\\
NOTE: Logic tables have $2^n$ rows where $n = $ the number of connectives!\\
\\
Definitions:\\
(1) Tautology: all values evaluate to true on the truth table\\
(2) Fallacy: all values evaluate to false on the truth table\\
(3) Contingency: a proposition that is not a tautology or a fallacy
\subsection{Laws of Prop Logic + Inference rules}
\includegraphics[width=8cm]{quicklatex.com-67ea85d519042c10897afc964c777539_l3.png}   
\includegraphics[width=10cm]{Screen Shot 2023-03-08 at 11.57.11 PM.png}
\subsection{First Order/Predicate Logic}
Predicates are propositions that depend on one or more variables (Prop Logic $\subset$ FOL first-order logic). \\
\\
EXAMPLE: $x+2 = 5;\;p(x,y);\;\text{even}(x,y);\;x+y > 10$ \\
\\
(1) $\forall$ (Universal quantifier): requires a variable name, requires a domain name $\rightarrow$ $\forall x \in A\;\;\;p(x,y)$\\
(2) $\exists$ (Existential quantifier): requires a variable name, requires a domain name $\rightarrow$ $\exists x \in A\;\;\;p(x,y)$\\
\\
EXAMPLE: There exists an even integer that is even and prime $\implies \exists x \in Z\;\;\;(2|x \wedge \text{prime}(x))$ \\
\\
NOTE: $\forall$ is associated with $\rightarrow$, $\exists$ associated with $\wedge$ $\Longrightarrow$ $\exists x \in A\;\;p(x) \wedge n(x)$ OR $\forall x \in A\;\;p(x) \rightarrow n(x)$ \\
$$
\neg(\exists x \in A\;\;p(x)) \equiv \forall x \in A\;\;\neg p(x) \quad \text{OR} \quad \neg(\forall x \in A\;\;p(x)) \equiv \exists x \in A\;\;\neg p(x)
$$
\pagebreak
\section{Proofs}
(1) Enumeration [exhaustive] $\rightarrow$ limited number of cases\\
(2) Direct $\rightarrow$ directly from some $p(x) \dots q(x)$\\
(3) Contrapositive $\rightarrow$ proving that the contrapositive is true (employs other strategies)\\
(4) IFF [both directions] $\rightarrow$ you need to prove both $p \rightarrow q \wedge q \rightarrow p$ \\ 
(5) Cases [similar to enumeration] $\rightarrow$ infinite domain such that $x$ is even vs. $x$ is odd\\ 
(6) Contradiction \\
(7) Counterexample [Disproof]\\
(8) Induction
\section{Sets}
\subsection{Definitions}
(1) Sets: a collection of distinct objects in which order does not matter (non-distinct sets are called multisets)\\
(2) Sequences: ordered sets, and collections are considered unordered
\subsection{Set Builder Notation}
(1) $\in$ denotes that an object $a$ is in a set $A \rightarrow a \in A$.\\
(2) $\{\}, \varnothing$ denotes the empty set ($\varnothing \subseteq A$)
(3) Subset is defined as every element of $A$ being in $B$. $A \subseteq B \iff \forall a \in A\;\;a \in B$.
$$
\{\} \notin \{1,2,3,4\}\quad\text{BUT}\quad\{\} \subseteq \{1,2,3,4\}
$$
$$
\{1\} \notin \{1,2,3,4\}\quad\text{BUT}\quad\{1\} \subseteq \{1,2,3,4\}
$$
$$
1 \in \{1,2,3,4\}\quad\text{BUT}\quad 1 \nsubseteq \{1,2,3,4\}
$$
$$
\{1,2\} \notin \{1,2,3,4\}\quad\text{BUT}\quad\{1,2\} \subseteq \{1,2,3,4\}
$$
$$
\{1,2\} \in \{\{1,2\},3,4\}\quad\text{BUT}\quad\{1,2\} \nsubseteq \{\{1,2\},3,4\}
$$
(3) \textbf{Equality of Sets:} $A = B \iff (A \subseteq B) \wedge (B \subseteq A)$ \\
(4) \textbf{The Universal Set} (the whole domain) is denoted $U$ which is the set of everything under consideration.
\subsection{Power Set, Cardinality, Finite Set, Cartesian Product}
(1) The set of all subsets of a set $S$ is called its \textbf{power set}. Note $P(\varnothing) = \{\varnothing\}$ and that $\varnothing \in P(S)$.
$$
S = \{1,2\} \implies P(S) = \{\varnothing,\{1\},\{2\},\{1,2\}\}
$$
For every set $S$, we know that $P(S)$ has $2^{n}$ elements where $n$ is the number of elements in $S$.\\
(2) A \textbf{finite set} is a set with $n \in \mathbb{Z}^+$ elements. Otherwise, the set is defined as infinite.
(3) The \textbf{cardinality} of a set $S$ is the number of elements of $S$. \\
(4) The \textbf{Cartesian Product} between sets is the set of pairs crossing each element from $a$ with each element from $b$. 
$$
A \times B = \{(a,b)\;|\;a \in A \wedge b \in B\}
$$
\subsection{Set Operations}
(1) $A \cap B$ Intersection $\rightarrow \{x\;|\;x \in A \wedge x \in B\}$\\
(2) $A\cup B$ Union $\rightarrow \{x\;|\;x \in A \vee x \in B\}$\\
(3) $A-B$ Difference $\rightarrow \{x\;|\;x \in A \wedge x \notin B\}$\\
(4) $A^c$ Complement $\rightarrow \{x\;|\;x \notin A\}$
\subsection{Set Properties}
(1) Commutability (applies for $\cup$ and $\cap$): 
$$
A \cup B = B \cup A
$$
(2) Associability (applies for $\cup$ and $\cap$):
$$
A \cup (B \cup C) = (A \cup B) \cup C
$$
(3) Distributivity (applies for $\cup$ and $\cap$):
$$
A \cap (B \cup C) = (A \cap B) \cup (A \cap C)
$$
(4) De Morgan's Law (applies for $\cup$ and $\cap$):
$$
\overline{A \cup B} = \overline{A} \cap \overline{B}
$$
$$
\overline{A \cap B} = \overline{A} \cup \overline{B}
$$
\pagebreak
\subsection{Proof With Sets}
(1) \textbf{Proving that $A \subseteq B$}. We need to show that $\forall a \in A\;\;\Big(a \in B\Big)$, proving that every element in one set is also in the other. \\
(2) \textbf{Proving that $A = B$}. You need to prove both directions of this statement such that $A \subseteq B$ and $B \subseteq A$.
\subsection{Useful Reference Problems}
(1) Construct the following set $A = \{1,6,11,16,21,\dots\}$
$$
A = \{x\;|\;x \in \mathbb{Z} \wedge (\forall n \in \mathbb{Z}\;(1\leq n \rightarrow x = 5n-4))\}
$$
\end{document}
