
\section{The Trout Problem}
\subsection{Analysis of Dynamical System}
We can represent the dynamical system with the following recursive definition where $a_n$ represents the trout population at year $n$. This follows from the recursive definition of dynamical systems with carrying capacities.
$$
a_n  = 1.7 \cdot a_{n-1} \cdot (1-\frac{a_{n -1}}{20000})
$$
In terms of fixed points, let's solve the equation:
$$
x = 1.7 \cdot x \cdot (1-\frac{x}{20000}) \implies x = \{0,\;8235.3\}
$$
We know that $f'(x) = 1.7\left(1-0.0001x\right)$. Thus, we just need to examine the behavior of this above function at our fixed points.
$$
f'(0) = 1.7 > 0 \implies \text{repelling fixed point}
$$
$$
f'(8235.3) = 0.3 > 0 \implies \text{repelling fixed point}
$$
\subsection{Natural Growth}
We want to find the number of years it will take for the trout population to reach 19,000 where our initial value is 10,000. We can do this recursively via a python program:
\begin{lstlisting}[language=Python]
count, trout = 0, 10000
while(trout < 19000):
    count += 1
    trout = 1.7*trout*(1-trout/20000)
return count
\end{lstlisting}
Interestingly enough, this produces a never-ending loop - which means that it never reaches 19k threshold.
\subsection{Supplemented Growth}
In terms of stocking $7000$ per year, we can think of this in terms of the exact same python equation (except slightly adjusted):
\begin{lstlisting}[language=Python]
count, trout = 0, 10000
while(trout < 19000):
    count += 1
    trout = 1.7*trout*(1-trout/20000) + 7000
return count
\end{lstlisting}
\end{document}
