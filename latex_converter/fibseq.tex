\section{Generating Functions & Modified Fibonacci}
\subsection{The Problem}
The Fibonacci numbers are defined by the recurrence $a_0 = a_1 = 1$ and $a_n = a_{n-1} + a_{n-2}$ for $n \geq 2$.
Consider the modified Fibonacci numbers defined by $b_0 = b_1 = 1$ and $b_n = b_{n-1} + 2b_{n-2}$ for $n \geq 2$ (so the first few terms are $1, 1, 3, 5, 11, 21, 43, 85, \ldots$).
Using generating functions, find a closed-form expression for the modified Fibonacci numbers $b_n$.\\
\\
The problem of finding a closed-form expression for the modified Fibonacci numbers using generating functions is fascinating because it extends the classic Fibonacci sequence into new mathematical territory.
The use of generating functions to tackle this problem showcases the power of algebraic methods in deriving explicit formulas from recursive definitions. This approach not only provides deeper insights into the properties of the sequence but also highlights the versatility and elegance of generating functions in solving complex recurrence relations. Exploring modified sequences like this one enriches our understanding of mathematical patterns and their applications in various fields, from theoretical research to practical problem-solving.
\subsection{A Formal Proof}
In this problem, we have the following modified version of the Fibonacci sequence defined by the following relation.
$$
F_n = F_{n-1} + 2F_{n-2} \quad \quad \quad F_0 = F_1 = 1
$$
The goal of this exercise is to define some generating function $F(x)$ as the formal power series whose coefficients are the modified Fibonacci numbers themselves. The reason why we want to set up the function this way is because we can use power series properties to deduce information about these modified Fibonacci numbers so that we can express it as a closed-form sequence. The formal definition of $F(x)$ is found below. 
\begin{align*}
F(x) &= \sum_{n=0}^\infty F_n \cdot x^n \\
&= 1 + \sum_{n=1}^\infty F_n \cdot x^n 
\end{align*}
If we separate out the first two terms, we are left with the following expression. Then, all we need to do is substitute the recursion relation to get the modified expression and the two sums of our interest $S_1$ and $S_2$.
\begin{align*}
F(x) &= 1 + x + \sum_{n=2}^\infty F_n \cdot x^n \\
&= 1 + x + \sum_{n=2}^\infty (F_{n-1} + 2F_{n-2}) \cdot x^n\\
&= F(x) = 1 + x + \underbrace{\sum_{n=2}^\infty F_{n-1}\cdot x^n}_{S_1} + \underbrace{\sum_{n=2}^\infty 2F_{n-2}\cdot x^n}_{S_2}
\end{align*}
Now, let's re-index the sums $S_1$ and $S_2$. First, let's do $S_1$. The goal of the re-indexing is to put the sum back in terms of the series definition of $F(x)$.
\begin{align*}
    \sum_{n=2}^\infty F_{n-1} \cdot x^{n} &= x\sum_{n=2}^\infty F_{n-1} \cdot x^{n-1} \\
    &= x \sum_{n=1}^\infty F_{n}\cdot x^n \\
    &= x\cdot (F(x) - 1)
\end{align*}
Now re-index the sum $S_2$ in a similar manner.
\begin{align*}
\sum_{n=2}^\infty 2F_{n-2}\cdot x^n &= 2 \sum_{n=2}^\infty F_{n-2} \cdot x^n \\
&= 2x^2 \sum_{n=2}^\infty F_{n-2} \cdot x^{n-2} \\
&= 2x^2 \sum_{n=0}^\infty F_n \cdot x^n \\
&= 2x^2 \cdot F(x) 
\end{align*}
Using our $S_1$ and $S_2$ rewrites, we can then rewrite our original expression for $F(x)$ as follows. 
\begin{align*}
F(x) &= 1 + x + \left(x \cdot (F(x) - 1)\right) + \left(2x^2 \cdot F(x)\right)\\
&= 1 + x \cdot F(x) + 2x^2 \cdot F(x)
\end{align*}
Solving for $F(x)$, we get the following expression, which we can factor using elementary algebra.
$$
F(x) = \frac{1}{1 - x - 2x^2} \implies F(x) = \frac{1}{(1-2x)(x+1)}
$$
Now, we want to do partial fraction decomposition to re-express $F(x) = \frac{\frac{2}{3}}{1 - 2x} + \frac{\frac{1}{3}}{1+x}$.
$$
\frac{1}{(1-2x)(x+1)} = \frac{A}{1-2x} + \frac{B}{x+1} \implies 1 = A(x+1) + B(1-2x) \implies A = \frac{2}{3} = , B = \frac{1}{3}
$$
We did this fraction decomposition in order to tell us more about the power series, where we can utilize the property that $\sum_{n=0}^\infty x^n = \frac{1}{1-x}$. As a result of the above, we can rewrite $F(x)$ as follows.
$$
F(x) = \underbrace{\frac{2}{3} \left(\frac{1}{1-2x}\right)}_{S_3} + \underbrace{\frac{1}{3} \left(\frac{1}{1+x}\right)}_{S_4}
$$
Now that our fractions are in the form of the sum of power series, let's first look at rewriting $S_3$. 
$$
S_3 = \frac{2}{3}\left(\frac{1}{1-\left(2x\right)}\right) = \frac{2}{3} \sum_{n=0}^\infty (2x)^n
$$
Similarly, let's rewrite at $S_4$. 
$$
S_4 = \frac{1}{3} \left(\frac{1}{1-(-x)}\right) = \frac{1}{3}\sum_{n=0}^\infty (-x)^n
$$
Using these rewrites, we are able to create expressions for $F_n$ that are not reliant on any recursive definition and only dependent on $n$. Thus, we can then express $F(x)$ as follows.
\begin{align*}
F(x) &= \overbrace{\frac{2}{3} \cdot \sum_{n=0}^\infty 2^n \cdot x^n}^{S_3} + \overbrace{\frac{1}{3} \cdot \sum_{n=0}^\infty (-1)^n\cdot  x^n}^{S_4} \\
&=  \sum_{n=0}^\infty \frac{2^{n+1}}{3} \cdot x^n + \sum_{n=0}^\infty \frac{(-1)^n}{3} \cdot x^n \\
&= \sum_{n=0}^\infty \underbrace{\left(\frac{2^{n+1}}{3} + \frac{(-1)^n}{3}\right)}_{F_n} \cdot x^n
\end{align*}
Remember from the beginning, we defined $F(x) = \sum_{n=0}^\infty F_n \cdot x^n$. Well, using these generating functions, we just figured out a way to quantify the coefficient $F_n$ to be closed-form (only reliant on $n$). 
$$
F_n = \frac{2^{n+1}}{3}  + \frac{(-1)^n}{3}
$$ 
This means that for all $n \geq 2$, we have the following formula for the $n$-th modified Fibonacci number $F_n$. To further prove that this formula works, I've listed a few of the first numbers of this sequence, which you can also check by the recursive relation.
$$
\begin{tabular}{l l}
    $F_2$ &$=\frac{2^{3}}{3}  + \frac{(-1)^2}{3} = 3$\\
    $F_3$ &$=\frac{2^{4}}{3}  + \frac{(-1)^3}{3} = 5$\\
    $F_4$ &$=\frac{2^{5}}{3}  + \frac{(-1)^4}{3} = 11$\\
    \multicolumn{1}{c}{$\vdots$} & \multicolumn{1}{c}{$\vdots$}\\
    \multicolumn{1}{c}{$\vdots$} & \multicolumn{1}{c}{$\vdots$}\\
    $F_n$ & $= \frac{2^{n+1}}{3}  + \frac{(-1)^n}{3}$
\end{tabular}
$$
This completes the proof. A similar process for how to define the original Fibonacci sequence is found \href{https://austinrochford.com/posts/2013-11-01-generating-functions-and-fibonacci-numbers.html}{here}.
\end{document}