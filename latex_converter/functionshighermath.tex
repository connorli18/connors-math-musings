\section{Functions}
\subsection{General Notes}

\textbf{Definition:} Take two sets, $X$ and $Y$. A function $f: X \to Y$ (a function from $X$ to $Y$) is a collection of ordered pairs $(x,y)$ such that $x \in X$ and $y \in Y$. \\
\\
\begin{itemize}
  \item The condition for these ordered pairs to be considered a function is that they follow the Vertical Line Test (VLT). Anoterh way to put this is that each input only ever has one output!
\end{itemize}
More formally, we can say that $\forall x \in X$, the function $(x,y)$ only ever has one $y$. In these scenarios, we tend to use the "negative" definition, and say that if $(x_1, y_1) \in f$ and $(x_1, y_2) \in f$, then $y_1 = y_2$. 

\subsection{Definitions}

Some important definitions are as follows:

\begin{itemize}
\item \textbf{Image (Range):} The image of $f: X \rightarrow Y$ is basically all the values in $Y$ that values $x \in X$ map to. In order words, it's the "range" of the function. 
  \begin{itemize}
    \item However, we use the term image because this refers to the "range" of a specific set. You can imagine that you have a function $f(x) = x^2$ defined over all the real numbers. The range is $(0, \infty)$, but let's say that we only cared about the inputs $\{1,2,3\}$. Then, the image of this set of the function would be $\{1,4,9\}$. 
  \end{itemize}
\item \textbf{PreImage (Domain):} This is a fancy word for domain, but it's basically like all the elements $x \in X$ that map to values $y \in Y$. In the above example, take something like $\{1,4,9\}$ as output values for the function $f(x) = x^2$. Then, we can say the preimage of this set is $\{1,2,3\}$ because those are the values that map to those outputs.
\end{itemize}

\subsection{Mapping Types}

There are three types of "mappings". Consider the function $f: A \to B$ where $a \in A$ and $b \in B$. 

\begin{itemize}
\item \textbf{Injective (one-to-one):} If $f(a) = f(b)$, then $a = b$. Another way to say this is that every input value only maps to one output value. You will never have a case where your function has something like $f(1) = f(3)$. 
\item \textbf{Surjective (onto):} For all $b \in B$, there exists some $a \in A$ such that $f(a) = b$. This is basically saying that every possible output (in set $B$) guarentees to have some input from set $A$ that can map to it. 
\item \textbf{Bijective:} Both surjective and injective! 
  \begin{itemize} 
    \item <u>NOTE</u>: If they ever ask about the inverse of a function, this is only possible with a bijective function! If you think about it, it makes sense why you can only have an inverse (that is a function) if the properties of injectivity and surjectivity are satisfied.
  \end{itemize}
\end{itemize}