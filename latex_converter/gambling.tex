\section{Basic Math Behind Sports Gambling}
Don't worry, guys...I can stop whenever I want, but I did want to share some thoughts about implied probability and odds related to sports gambling. 

\subsection{Odds & Implied Probability}
Some basic terminology, oftentimes you will see "moneylines" on games like PSG vs. Bayern that will look like this.
\begin{itemize}
    \item PSG (+360)
    \item Bayern (-460)
\end{itemize}
What this means is for the (+), you bet \$100 to earn \$360 dollars. For the (-), you must bet \$460 dollars to earn \$100. 
This "implies" that your probability for winning are as such. Define some $A$ as the line (+) and $B$ as the moneyline for the (-). \\
\\
Let's focus on $A$. 
If we assume that this line is fair or that the sportsbook wants to make it as realistic as possible, we can assume that the 
$$
\text{Probability of Winning} \times \text{Winnings} = \text{Probability of Loss} \times \text{Losses}
$$
By definition of the positive "moneyline" and assigning some prbability of winning as $w_p$, we have that
$$
w_p \cdot A = 100 \cdot (1-w_p) \implies w_p = \frac{100}{A + 100}
$$
Similarly, we can do the same thing with the negative moneyline, and we'll call that probaiblity $w_n$ .
$$
w_n \cdot 100 = (1-w_n) \cdot B \implies w_n = \frac{B}{B + 100}
$$
\subsection{Arbitrage?}
Usually, sportsbooks set their odds so that the combined implied probabilities of events are less than $1$. However, if you bet across bookies or sportsbooks and are able to generate the total implied probability $< 1$, then you can earn free money!
