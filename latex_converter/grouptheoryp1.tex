\section{A Problem on Group Theory}
\subsection{Top-Level Problem}
Let $G$ denote the set of $3 \times 3$ matrices of the form 
$$
\begin{pmatrix}
    a & b & e\\
    c & d & f \\
    0 & 0 & \lambda
\end{pmatrix}
$$
that satisfy the relation $(ad - bc)\lambda = 1$.
\subsection{Part (a)}
\subsubsection{Problem}
(a) Show that $G$ is a group.
\subsubsection{Solution}

Let's prove the properties of groups under matrix multiplication.\\
\\
\underline{[1] Closure}. Take any two elements $M_1,M_2 \in G$, we know that $M_1 \cdot M_2 \in G$.
$$
\begin{pmatrix}
    a_1 & b_1 & e_1\\
    c_1 & d_1 & f_1\\
    0 & 0 & \lambda_1
\end{pmatrix}
\begin{pmatrix}
    a_2 & b_2 & e_2\\
    c_2 & d_2 & f_2\\
    0 & 0 & \lambda_2
\end{pmatrix} = \begin{pmatrix}
    a_1a_2 + b_1c_2 & a_1b_2 + b_1d_2 & a_1e_2 + b_1f_2 + e_1\lambda_2\\
    a_2c_1 + c_2d_1 & b_2c_1+ d_1d_2 & c_1e_2 + d_1f_2 + f_1\lambda_2 \\
    0 & 0 & \lambda_1 \lambda_2
\end{pmatrix}
$$
\begin{align*}
((a_1a_2 + b_1c_2)(b_2c_1+ d_1d_2) - (a_1b_2 + b_1d_2)(a_2c_1 + c_2d_1))(\lambda_1 \lambda_2) &= 1\\
((a_1d_1 - c_1b_1)(a_2d_2-c_2b_2))(\lambda_1 \lambda_2) &= 1\\
(a_1d_1 - c_1b_1)(\lambda_1)\cdot (a_2d_2-c_2b_2)(\lambda_2) &= 1
\end{align*}
\underline{[2] Associative}. Take any three elements $M_1,M_2,M_3 \in G$. We know that these matrices are associative by a property from linear algebra that matrix multiplication is associative.\\
\\
\underline{[3] Identity}. The identity element of any $n \times n$ matrix is the diagonal $I_n$, another property from linear.\\
\\
\underline{[4] Inverse}. Applying our formula for inverse of $3 \times 3$, we have that the matrix for any $M \in G$ is $M^{-1}$.
$$
M^{-1} = \begin{pmatrix}
    d\lambda & -b\lambda & bf-ed\\
    -c\lambda & a\lambda & ec - af\\
    0 & 0 & \frac{1}{\lambda}
\end{pmatrix}
$$
We know that this $M^{-1} \in G$ by the following.
$$
\lambda^2(ad - bc)\cdot \frac{1}{\lambda} = (ad - bc)\lambda = 1
$$


\subsection{Part (b)}
\subsubsection{Problem}
(b) Show that the subset $H \subset G$ for which $a = d = 1$ and $b = c = 0$ is a subgroup.
\subsubsection{Solution}

Let's prove the subgroup properties. Since $H$ is non-empty and definitionally $H \subset G$, we just have to prove closure under products, inverse, and identity. The identity is trivial since the scenario where $a=d=\lambda = 1$ and $c=b=e=f=0$ is the identity, $I_3 \in H$.\\
\\
Another property of group $H \subset G$ is that $(ad - bc)\lambda = 1 \implies \lambda = 1$. For closure, take any two elements $M_1, M_2 \in H$. From below, we can see since $e_i + f_i \in \mathbb{R}$, then $M_1 \cdot M_2 \in H \subset G$.
\begin{align*}
    \begin{pmatrix}
        1 & 0 & e_1 \\
        0 & 1 & f_1 \\
        0 & 0 & 1
    \end{pmatrix}    
    \begin{pmatrix}
        1 & 0 & e_2 \\
        0 & 1 & f_2 \\
        0 & 0 & 1
    \end{pmatrix} = \begin{pmatrix}
        1 & 0 & e_1 + e_2 \\
        0 & 1 & f_1 + f_2 \\
        0 & 0 & 1
    \end{pmatrix}
\end{align*}
For inverse, we can use a formula from linear algebra to show that $M \in H$ emits an inverse $M^{-1}$ defined below and $M^{-1} \in H \subset G$.
$$
M^{-1} = \begin{pmatrix}
        1 & 0 & -e \\
        0 & 1 & -f \\
        0 & 0 & 1
    \end{pmatrix} 
$$



\subsection{Part (c)}
\subsubsection{Problem}
(c) Show that the defined $H$ in Part (b) is a normal subgroup of $G$.
\subsubsection{Solution}

We simply have to prove that $gHg^{-1} = H\; \forall g \in G$. All elements of $H$ can be expressed as a matrix with $a=d=\lambda = 1$ and $c=b=0$ where $e_0,f_0 \in \mathbb{R}$ are arbitrary. Now, consider any arbitrary $g \in G$.
\begin{align*}
    \begin{pmatrix}
        a & b & e\\
        c & d & f\\
        0 & 0 & \lambda
    \end{pmatrix}
    \begin{pmatrix}
        1 & 0 & e_0 \\
        0 & 1 & f_0 \\
        0 & 0 & 1
    \end{pmatrix}
    \begin{pmatrix}
        d\lambda & -b\lambda & bf-ed\\
        -c\lambda & a\lambda & ec - af\\
        0 & 0 & \frac{1}{\lambda}
    \end{pmatrix} &= \begin{pmatrix}
        \lambda(ad-bc) & 0 & -ead\lambda + ae_0 + ebc\lambda + bf_0 + e\\
        0 & \lambda(ad-bc) & -adf\lambda + bcf\lambda + ce_0 + df_0 + f\\
        0 & 0 & 1
    \end{pmatrix}\\
    &= \begin{pmatrix}
        1 & 0 & -ead\lambda + ae_0 + ebc\lambda + bf_0 + e\\
        0 & 1 & -adf\lambda + bcf\lambda + ce_0 + df_0 + f\\
        0 & 0 & 1
    \end{pmatrix}\\
    &= \begin{pmatrix}
        1 & 0 & ae_0 + bf_0\\
        0 & 1 & ce_0 + df_0\\
        0 & 0 & 1
    \end{pmatrix}
\end{align*}
Since every $m,n \in \mathbb{R}$ can be expressed as linear combination of $ae_0 + bf_0, ce_0 + df_0$ with fixed $a,b,c,d$, we know that $gHg^{-1} = H \;\forall g \in G$.


\subsection{Part (d)}
\subsubsection{Problem}
(d) Let $\phi: G \to GL(2,\mathbb{R})$ be defined as the following map.
$$
\phi\left(\begin{pmatrix} a & b & e \\ c & d & f \\ 0 & 0 & \lambda \end{pmatrix}\right) = \begin{pmatrix} a& b \\ c & d\end{pmatrix}
$$
Show that $\phi$ is a homomorphism and that $\phi(g)$ is the identity iff $g \in H$.
\subsubsection{Solution}
First, we know that $\phi(g)$ is well-defined for all $g$. We know that $\phi(g) \in GL(2,\mathbb{R})$ and $\phi(g)$ is unambiguous for every $g \in G$. We also know that if $g_1 = g_2 \implies \phi(g_1) = \phi(g_2)$, which completes our proof of well-definedness. \\
\\
Now, to prove homomorphism, we must show that $\phi(g_1 \cdot g_2) = \phi(g_1) \cdot \phi(g_2)$. To do this, consider any arbitrary $g_1, g_2 \in G$.
$$
g_1 \cdot g_2 = \begin{pmatrix}
    a_1 & b_1 & e_1\\
    c_1 & d_1 & f_1\\
    0 & 0 & \lambda_1
\end{pmatrix}
\begin{pmatrix}
    a_2 & b_2 & e_2\\
    c_2 & d_2 & f_2\\
    0 & 0 & \lambda_2
\end{pmatrix} = \begin{pmatrix}
    a_1a_2 + b_1c_2 & a_1b_2 + b_1d_2 & a_1e_2 + b_1f_2 + e_1\lambda_2\\
    a_2c_1 + c_2d_1 & b_2c_1+ d_1d_2 & c_1e_2 + d_1f_2 + f_1\lambda_2 \\
    0 & 0 & \lambda_1 \lambda_2
\end{pmatrix}
$$
$$
\phi(g_1g_2) = \begin{pmatrix}
    a_1a_2 + b_1c_2 & a_1b_2 + b_1d_2 \\
    a_2c_1 + c_2d_1 & b_2c_1+ d_1d_2 
\end{pmatrix} = \begin{pmatrix}
    a_1 & b_1 \\
    c_1 & d_1 
\end{pmatrix} \begin{pmatrix}
    a_2 & b_2 \\
    c_2 & d_2 
\end{pmatrix} = \phi(g_1) \cdot \phi(g_2)
$$
Thus, we have proven that $\phi$ is a valid homomorphism. Now, if $g \in H$, then we know $a = d = 1$ and $b = c = 0$, which means that mapping $\phi(g)$ creates the identity matrix $I_2$. Similarly, if $\phi(g) = I_2$, then we must know that $a = d = 1$ and $b = c = 0$ by uniqueness of the identity, which means that $g \in H$ by definition.