\section{Ambiguous Context Free Grammar}
\subsection{Problem Statement}
Given three languages $L_1, L_2, L_3$ over the same alphabet $\Sigma$, we write $\text{MAJ}(L_1, L_2, L_3)$ to denote the following language.
$$
\text{MAJ}(L_1, L_2, L_3) = \{w \in \Sigma \;:\; w \text{ belongs to at least two of } L_1, L_2, L_3 \}
$$
If $L_1, L_2, L_3$ are all regular, is $\text{MAJ}(L_1, L_2, L_3)$ always regular?
\subsection{A Formal Proof}
Yes, it is always regular. We know that $\text{MAJ}$ can only accept a string $w$ if $w$ is accepted in one of the following languages:
\begin{itemize}
    \item $L_1 \cap L_2$
    \item $L_2 \cap L_3$
    \item $L_1 \cap L_3$
    \item $L_1 \cap L_2 \cap L_3$
\end{itemize}
Another way to write this is that the language $\text{MAJ}$ is the union of these four possibilities listed above:
$$
\text{MAJ}(L_1,L_2,L_3) = (L_1 \cap L_2) \cup (L_1 \cap L_3) \cup (L_2 \cap L_3) \cup (L_1 \cap L_2 \cap L_3)
$$
Since the union of regular languages also produces a regular language, we now only have to show that the individual components (the four listed above) are also regular in order to prove that $MAJ$ is regular. 
To do this, we must prove the lemma that the intersection of two regular languages is also regular.\\
\\
Using De Morgan's Law, we can state that $L_1 \cap L_2 = \overline{\overline{L_1} \cup \overline{L_2}}$. Since the complement of regular languages is regular, we know that both $\overline{L_1}$ and $\overline{L_2}$ are regular. And applying the union rule above for regular languages, we can conlude that $\overline{L_1} \cup \overline{L_2}$ is also a regular language and that its complement, $\overline{\overline{L_1} \cup \overline{L_2}}$, is regular.  Thus, we know that $L_1 \cap L_2$ is regular. \\
\\
Since the intersection of two regular languages is regular, then we can conclude that each of the four components above is regular, which means their union (composing $\text{MAJ}$) is also regular. Thus, $\text{MAJ}$ is regular.