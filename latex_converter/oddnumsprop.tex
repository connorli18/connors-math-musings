\section{Interesting Sum of Odd Numbers}
\subsection{The Proposition}
Consider the following theorem.\\
\\
\textbf{Proposition:} The sum of the first $n$ odd integers is a perfect square. Specifically, this sum of interest is $n^2$. 

\subsection{Proof Method (1): Visual Proof}
Consider the case where $n = 1$. Visually, we can describe this as a $1 \times 1$ square (represented by the matrix).
$$
\begin{pmatrix}
    1
\end{pmatrix}
$$
Now, consider the case where $n = 2$. Since the second odd number is $3$, we can represent this still in square form as follows (where the cells with $2$ represent the contribution from $n =2$).
$$
\begin{pmatrix}
    1 & 2 \\
    2 & 2
\end{pmatrix}
$$
If we extend this to $n = 3$, we can represent this as follows.
$$
\begin{pmatrix}
    1 & 2 & 3 \\
    2 & 2 & 3 \\
    3 & 3 & 3 
\end{pmatrix}
$$
For the case with any $n \in \mathbb{Z}$, we can represent this (still) in square form.
$$
\begin{pmatrix}
    1 & 2 & \cdots & n-1 & n \\
    2 & 2 & \cdots & n-1 & n \\
    \vdots & \vdots & \ddots & \vdots & \vdots \\
    n-1 & n-1 & \cdots & n-1  & n-1 \\
    n & n & \cdots & n & n
\end{pmatrix}
$$
\subsection{Proof Method (2): Induction}
Consider the base case $n=1$. It should be fairly obvious that $1 = 1^2$.\\
\\
Now, assume the following inductive hypothesis.\\
\\
\textbf{Inductive Hypothesis:} The proposition holds for $n$ such that $n^2 = 1 + \cdots + (2n-1)$.\\
\\
Now, consider the case for $n +1$. 
\begin{align*}
1 + \cdots + (2n -1) + (2n + 1) &= n^2 + 2n + 1 \\
&= (n+1)^2 
\end{align*}
\subsection{Proof Method (3): Gaussian Sum}
We can use the Gaussian sum trick to show this directly.
$$
\frac{\left(1 + (2n-1)\right) * n}{2} = \frac{2n^2}{2} = n^2
$$