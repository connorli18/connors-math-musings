\section{DFA Construction & Language Regularity}
\subsection{Problem Statement}
Let $L$ be a language consisting of all strings $a$'s and $b$'s with an equal number of $ab$ and $ba$ as substrings. Is $L$ a regular language?
\subsection{A Formal Proof}
For the language described, to prove that it is regular, all we need to do is construct a valid DFA. For the language $L$, let's call its respective DFA $D$. To construct $D$, let's begin by defining the components of the DFA $D(Q,\Sigma,\delta,q_0,F)$.
\begin{enumerate}
    \item $Q:$ We will use 5 states labeled $\{q_0,\;\dots\;,q_4\}$ detailed in the description below
    \item $\Sigma:$ $\{a,b\}$ as defined by the problem
    \item $\delta:$ transition function will be outlined in more detail in the diagram below
    \item $q_0:$ the initial state
    \item $F:$ set of $\{q_0,q_1,q_2\}$ outlined in the diagram below.
\end{enumerate}
The transition diagram is as follows:
\begin{center}
    \includegraphics[width=10cm]{Screen Shot 2023-02-17 at 2.35.09 PM.png}
\end{center}
\begin{itemize}
    \item State $q_0$ serves as an accept state to accept the empty string
    \item State $q_1$ is the accept state for strings beginning with $a$, utilizing the state $q_4$ to cycle between equal and not equal substrings of $ab,ba$. 
    \item States $q_2,q_3$ are the mirror of the left side of the diagram except they consider strings that start with $b$ - bouncing between equal and not equal substrings of $ba,ab$
\end{itemize}
Essentially, $q_0$ accepts the empty string in $L$. Next, if the string starts with $a$, the transition function takes you to the left part of the DFA, and if the string starts with $b$, the transition function takes you to the right part of the DFA. There, you bounce between the two states (either $q_1,q_4$ pr $q_2,q_3$) as the substring amounts of $ab,ba$ change. And since this a valid DFA for the language, we know that the language it represents is regular.