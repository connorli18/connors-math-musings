\section{Infinitely Many Primes of Form $3k+2$}
\subsection{The Problem}
Prove that there are infinitely many primes of the form $3k+2$ or $p \equiv -1 \text{ mod }3$ where $k \in \mathbb{Z}^+$.
\subsection{A Formal Proof}
For the sake of contradiction, assume that there are a finite number of primes of form $p_i \equiv -1 \text{ mod } 3$. Enumerate all such primes up to the largest prime $p_n \equiv -1 \text{ mod } 3$.
$$
p_1 = 2,\;p_2 = 5,\; \dots\;,\; p_n
$$
Now, excluding $p_1$, define some $N$ as the product of such primes $p_i$ and $3$ plus $2$. 
$$
N = 3\cdot \left(\prod_{i=2}^n p_i \right) + 2
$$
Since we know that $p_i \nmid N$ for all $i \in [1,n]$, we can conclude that $N$ is one of two cases. Either $N$ is a prime of form $N \equiv -1 \text{ mod } 3$ or $N$ is divisible by some $p_m$ such that $m > n$. If $N$ is a prime, then we have reached a contradiction and our proof is complete.\\
\\
If $N$ is not a prime, then we know it is divisible by some prime $p_m$. We know that all primes larger that $p_n$ must be in the form of either $p_m = 1 + 3\ell$ or $p_m = 2 + 3\ell$. If there exists a prime divisor of the latter form, then we have a contradiction (a larger prime of the desired form) and our proof is complete, so consider the alternative scenario.\\
\\
If every divisor of $N$ is of form $p_k = 1 + 3\ell$ where $\ell \in \mathbb{Z}^+$, then $N$ must also be in the form $1+3\ell_0$, which is impossible since $N$ is, by definition, in form $N = 2 + 3\ell$. As shown below, prime multiplication of the form $p_k = 1+3\ell$ is closed under multiplication.
$$
\left(1 + 3\ell_1\right)\left(1 + 3\ell_2\right) = 1 + 3(\ell_1 + \ell_2 + 3\ell_1\ell_2) = 1 + 3\ell_3
$$
Thus, since $N$ cannot only be composed of factors $p_k = 1 + 3\ell$, then we have proven that it is either a prime of desired form or is divisible by a prime of desired form (which contradicts the assumption of finite primes).