\section{Real-Valued Dirichlet Characters}
\subsection{Problem Statement}
If $q$ is an integer such that every Dirichlet character $\text{ mod }q$ is real-valued, show that $q$ must divide $24$.
\subsection{A Short Proof}
Take any $q \in \mathbb{N}$ and analyze the multiplicative group $\left(\mathbb{Z}/q\mathbb{Z}, \times\right)$. By the Chinese Remainder Theorem, we have that $\left(\mathbb{Z}/q\mathbb{Z}, \times\right) \cong \left(\mathbb{Z}/p_1^{e_1}\mathbb{Z}, \times\right) \times \left(\mathbb{Z}/p_2^{e_2}\mathbb{Z}, \times\right) \times \cdots \times \left(\mathbb{Z}/p_k^{e_k}\mathbb{Z}, \times\right)$ for some prime factorization $q = p_1^{e_1}p_2^{e_2}\cdots p_k^{e_k}$. We will use the property that groups with only elements of order $2$ or less can satisfy the condition that all Dirichlet characters take on exclusively real values.\\
\\
Let's look at the subgroup $\left(\mathbb{Z}/p_i^{e_i}\mathbb{Z}, \times\right)$. Since each $\chi$ must be real-valued (i.e. non-trivial elements map only to $\pm 1$), we know that each group must have all non-trivial elements of order $2$ or less in order to be mapped exclusively to roots of unity of order $2$.
For any $p_i > 2$, we have that $\left(\mathbb{Z}/p_i^{e_i}\mathbb{Z}, \times\right)$ has $\phi(p_i^{e_i}) = p_i^{e_i-1}(p_i-1)$ non-trivial elements. 
If $e_i = 1$, then $\left(\mathbb{Z}/p_i^{e_i}\mathbb{Z}, \times\right)$ is cyclic and has some element of order $p_i-1$ by Lagrange's Theorem of finite groups, so the only realizable $p_i = 3$ since every other $p_i$ has some element that can be mapped to a root of unity of order greater than $2$. And if $e_i \geq 2$, then Cauchy's theorem tells us that for every prime $p_i$ dividing the order of the group, then there exists an element of order $p_i$. 
This means that for all $p_i > 2$ and $e_i > 1$, there exists some element of order $\neq 2$ in $\left(\mathbb{Z}/p_i^{e_i}\mathbb{Z}, \times\right)$, so all such groups are not realizable in context of the problem's conditions.\\
\\
Now, let's examine the case where $p_i = 2$. 
If we examine $e_i = 1$, this group only contains trivial elements. 
For $e_i = 2$, the multiplicative group of units is $\{1, 3\}$, both of order $2$. 
For $e_i = 3$, the multiplicative group of units is $\{1, 3, 5, 7\}$, all of order $2$. 
For $e_i \geq 4$, the element $3$ is guaranteed to be of higher order than $2$ since $2^n > 3^2$, which means that all subgroups $\left(\mathbb{Z}/2^{e_i}\mathbb{Z}, \times\right)$ where $e_i > 3$ are not realizable in the given conditions.\\ 
\\
Combining the analysis from above with the Chinese Remainder Theorem, we have that the only prime powers that satisfy the requirement for all elements in their multiplicative group to be order $2$ are $2^1, 2^2, 2^3, 3^1$. Given the constraints, the largest possible product of prime factors that satisfies the requirements is $2^3 \cdot 3$ or $24$. Thus, all $\chi$ have exclusively real values for $q | 24$. 